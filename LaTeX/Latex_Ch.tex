\documentclass{article}
\usepackage[utf8]{inputenc}
\usepackage{multicol}
\usepackage{amsmath}
\usepackage{graphicx}

\title{Challenging Problem}
\author{Parvez Alam : AI21RESCH01005 }
\date{April 2021}

\begin{document}

\maketitle
\begin{multicols}{2}
\section{Problem}
\textbf{Two points are chosen at random on a line of unit length. What is the probability that the three line segments so formed will have a length greater than 1/4 ?} \\
\textbf{Solution: }
Let the line segment be PQ and the points be A, B. \\
let the length of PA be x, and length of AB be y then 

   length of BQ &=1-(x+y) 
\begin{align}
    x &>0.25 \nonumber \\
    y &>0.25 \nonumber \\
    x+y &<0.75 \nonumber 
\end{align}
\includegraphics[width=\columnwidth]{c1.png}
Favourable cases=area of the above triangle=\(\frac{1}{32}\)  \\ 
Sample space will be represented by
\[x+y<1\]
\includegraphics[width=\columnwidth]{c2.png}
sample space = area of the above triangle=\(\frac{1}{2}\) 
\begin{align}
    Probability &=\frac{\frac{1}{32}}{\frac{1}{2}} \nonumber\\
                &=\frac{1}{32}\times 2 \nonumber \\
                &=\frac{1}{16} \nonumber
\end{align}





\end{multicols}
\end{document}
